\documentclass[a4paper,12pt]{report}

\usepackage{alltt, fancyvrb, url}
\usepackage{graphicx}
\usepackage[utf8]{inputenc}
\usepackage{float}
\usepackage{hyperref}
\usepackage{tikz}

% Questo commentalo se vuoi scrivere in inglese.
\usepackage[italian]{babel}

\usepackage[italian]{cleveref}

\title{Progetto di Programmazione ad Oggetti\\``The Exiled''}

\author{Luca Casadei, Francesco Pazzaglia, Marco Magnani, Manuel Baldoni}
\date{\today}


\begin{document}

\maketitle

\tableofcontents

\chapter{Analisi}

\section{Requisiti}
L'applicazione è un gioco che presenta un personaggio controllato dal giocatore con la possibilità di muoversi nella mappa in 4 direzioni e di combattere contro dei nemici utilizzando magie elementali. I nemici sconfitti potrebbero rilasciare delle cure o dei potenziamenti che favoriscono il giocatore. Per concludere il gioco, il giocatore deve entrare in possesso di 4 cristalli che vengono consegnati una volta sconfitti i 4 boss del gioco che sono nemici più difficili da sconfiggere.
\subsection*{Requisiti funzionali}
\begin{itemize}
    \item Movimento del giocatore.
    \item Movimento dei nemici.
    \item Presenza di oggetti ottenibili dal giocatore (Cure, potenziamenti e cristalli).
    \item Terminazione del gioco una volta raccolti 4 cristalli o se il giocatore viene sconfitto.
    \item Posizionamento e distribuzione degli oggetti.
    \item Possibilità di fare battaglie tra giocatore e nemici.
    \item Nemici più forti (boss) che se sconfitti rilasciano i cristalli per terminare il gioco.
    \item Possibilità di utilizzo di magie in battaglia di diverso tipo (Fuoco, Acqua, Fulmine, Erba).
    \item Aumento di livello tramite guadagno di esperienza sconfiggendo i nemici.
\end{itemize}

\subsection*{Requisiti non funzionali}
\begin{itemize}
    \item Il gioco deve essere multipiattaforma.
    \item \textit{TODO}
\end{itemize}

\section{Analisi e modello del dominio}

All'inizio al giocatore viene chiesto di che tipo elementale sarà il personaggio da controllare, in base a questa scelta gli verrà quindi assegnata una mossa di base di quel tipo e avrà le mosse di quel tipo potenziate fino alla fine del gioco.\\
Il giocatore può muoversi di una cella alla volta in una mappa a griglia in 4 direzioni (Nord, Est, Sud, Ovest) e una battaglia con i nemici (anch'essi di un certo tipo elementale) inizia quando la cella in cui è il giocatore combacia con quella di un nemico.\\
Per affrontare un nemico in battaglia il giocatore ha a disposizione massimo 4 mosse (non per forza dello stesso tipo del giocatore). Il combattimento avviene a turni alternati, il primo è il giocatore, poi il nemico e così via finché uno dei due viene sconfitto, se è il giocatore, il gioco termina.\\
Allo sconfiggere dei nemici viene conferita al giocatore una certa quantità di esperienza che serve ad aumentare di livello e un oggetto casuale, l'oggetto viene salvato nell'inventario del giocatore e potrà essere di diverso tipo, es. oggetto curativo(ripristina un tot di vita), booster di statistiche(per esempio aumenta l'attacco di un certo tipo di mosse) ecc.. L'aumento di livello comporta un incremento generale delle statistiche ovvero attacco, difesa e vita di un valore costante. Ogni 5 livelli verrà presentata al giocatore la possibilità di imparare una nuova mossa di un tipo casuale, se il giocatore ha già 4 mosse potrà decidere di scambiare quella nuova con una di quelle che conosce già. All'aumentare di livello sarà richiesta sempre più esperienza per passare a quello successivo.\\
Ci sono diverse classi di nemico, alcuni più deboli e altri più forti, come i boss, che se sconfitti consegnano uno dei 4 oggetti necessari per concludere il gioco (i cristalli).\\

\subsection{Tipi elementali}
I seguenti sono i tipi elementali presenti, ognuno è efficace rispetto ad un altro, quindi quando vengono usate mosse di un certo tipo che risulta essere efficace rispetto al tipo di un nemico, si avranno dei moltiplicatori del danno. Lo stesso vale per i nemici che usano mosse di un tipo efficace rispetto a quello scelto dal giocatore all'inizio.
\\\\
Ad esempio, nel seguente schema si vede che Fulmine è efficace contro Acqua.


\begin{center}
\begin{tabular}{ c c c }
    Fulmine & $\rightarrow$ & Acqua \\
    Erba & $\rightarrow$ & Fulmine \\
    Acqua & $\rightarrow$ & Fuoco \\
    Fuoco & $\rightarrow$ & Erba \\
\end{tabular}
\end{center}

\subsection{UML}
\begin{figure}[H]
	\centering
	% generated by Plantuml 1.2024.2       
\definecolor{plantucolor0000}{RGB}{0,0,0}
\definecolor{plantucolor0001}{RGB}{241,241,241}
\definecolor{plantucolor0002}{RGB}{24,24,24}
\definecolor{plantucolor0003}{RGB}{235,147,127}
\definecolor{plantucolor0004}{RGB}{180,167,229}
\definecolor{plantucolor0005}{RGB}{173,209,178}
\scalebox{0.45}{
\begin{tikzpicture}[yscale=-1
,pstyle0/.style={color=black,line width=1.5pt}
,pstyle1/.style={color=plantucolor0002,fill=plantucolor0001,line width=0.5pt}
,pstyle2/.style={color=plantucolor0002,fill=plantucolor0003,line width=1.0pt}
,pstyle3/.style={color=plantucolor0002,line width=0.5pt}
,pstyle4/.style={color=plantucolor0002,fill=plantucolor0004,line width=1.0pt}
,pstyle6/.style={color=plantucolor0002,line width=1.0pt}
,pstyle7/.style={color=plantucolor0002,fill=plantucolor0002,line width=1.0pt}
]
\draw[pstyle0] (8.5pt,6pt) -- (134.9783pt,6pt) arc(270:360:3.75pt)  -- (144.4783pt,29.7461pt) -- (974.5pt,29.7461pt) arc(270:360:2.5pt)  -- (977pt,718pt) arc(0:90:2.5pt)  -- (8.5pt,720.5pt) arc(90:180:2.5pt)  -- (6pt,8.5pt) arc(180:270:2.5pt) ;
\draw[pstyle0] (6pt,29.7461pt) -- (144.4783pt,29.7461pt);
\node at (10pt,8pt)[below right,color=black]{\textbf{unibo.exiled.model}};
\draw[pstyle0] (301.5pt,183pt) -- (369.3414pt,183pt) arc(270:360:3.75pt)  -- (378.8414pt,206.7461pt) -- (644.5pt,206.7461pt) arc(270:360:2.5pt)  -- (647pt,444.5pt) arc(0:90:2.5pt)  -- (301.5pt,447pt) arc(90:180:2.5pt)  -- (299pt,185.5pt) arc(180:270:2.5pt) ;
\draw[pstyle0] (299pt,206.7461pt) -- (378.8414pt,206.7461pt);
\node at (303pt,185pt)[below right,color=black]{\textbf{character}};
\draw[pstyle0] (291.5pt,485.5pt) -- (323.2429pt,485.5pt) arc(270:360:3.75pt)  -- (332.7429pt,509.2461pt) -- (414.5pt,509.2461pt) arc(270:360:2.5pt)  -- (417pt,694pt) arc(0:90:2.5pt)  -- (291.5pt,696.5pt) arc(90:180:2.5pt)  -- (289pt,488pt) arc(180:270:2.5pt) ;
\draw[pstyle0] (289pt,509.2461pt) -- (332.7429pt,509.2461pt);
\node at (293pt,487.5pt)[below right,color=black]{\textbf{item}};
\draw[pstyle0] (251.5pt,50pt) -- (300.3909pt,50pt) arc(270:360:3.75pt)  -- (309.8909pt,73.7461pt) -- (362.5pt,73.7461pt) arc(270:360:2.5pt)  -- (365pt,147.5pt) arc(0:90:2.5pt)  -- (251.5pt,150pt) arc(90:180:2.5pt)  -- (249pt,52.5pt) arc(180:270:2.5pt) ;
\draw[pstyle0] (249pt,73.7461pt) -- (309.8909pt,73.7461pt);
\node at (253pt,52pt)[below right,color=black]{\textbf{combat}};
\draw[pstyle0] (673.5pt,347pt) -- (707.5545pt,347pt) arc(270:360:3.75pt)  -- (717.0545pt,370.7461pt) -- (812.5pt,370.7461pt) arc(270:360:2.5pt)  -- (815pt,583pt) arc(0:90:2.5pt)  -- (673.5pt,585.5pt) arc(90:180:2.5pt)  -- (671pt,349.5pt) arc(180:270:2.5pt) ;
\draw[pstyle0] (671pt,370.7461pt) -- (717.0545pt,370.7461pt);
\node at (675pt,349pt)[below right,color=black]{\textbf{move}};
\draw[pstyle0] (32.5pt,183pt) -- (61.6231pt,183pt) arc(270:360:3.75pt)  -- (71.1231pt,206.7461pt) -- (272.5pt,206.7461pt) arc(270:360:2.5pt)  -- (275pt,450pt) arc(0:90:2.5pt)  -- (32.5pt,452.5pt) arc(90:180:2.5pt)  -- (30pt,185.5pt) arc(180:270:2.5pt) ;
\draw[pstyle0] (30pt,206.7461pt) -- (71.1231pt,206.7461pt);
\node at (34pt,185pt)[below right,color=black]{\textbf{map}};
\draw[pstyle1] (831pt,388pt) arc (180:270:5pt) -- (836pt,383pt) -- (955.8651pt,383pt) arc (270:360:5pt) -- (960.8651pt,388pt) -- (960.8651pt,426pt) arc (0:90:5pt) -- (955.8651pt,431pt) -- (836pt,431pt) arc (90:180:5pt) -- (831pt,426pt) -- cycle;
\draw[pstyle2] (846pt,399pt) ellipse (11pt and 11pt);
\node at (846pt,399pt)[]{\textbf{\Large E}};
\node at (860pt,390.127pt)[below right,color=black]{ElementalType};
\draw[pstyle3] (832pt,415pt) -- (959.8651pt,415pt);
\draw[pstyle3] (832pt,423pt) -- (959.8651pt,423pt);
\draw[pstyle1] (396pt,224pt) arc (180:270:5pt) -- (401pt,219pt) -- (528.6315pt,219pt) arc (270:360:5pt) -- (533.6315pt,224pt) -- (533.6315pt,262pt) arc (0:90:5pt) -- (528.6315pt,267pt) -- (401pt,267pt) arc (90:180:5pt) -- (396pt,262pt) -- cycle;
\draw[pstyle4] (411pt,235pt) ellipse (11pt and 11pt);
\node at (411pt,235pt)[]{\textbf{\Large I}};
\node at (425pt,226.127pt)[below right,color=black]{\textit{GameCharacter}};
\draw[pstyle3] (397pt,251pt) -- (532.6315pt,251pt);
\draw[pstyle3] (397pt,259pt) -- (532.6315pt,259pt);
\draw[pstyle1] (315pt,388pt) arc (180:270:5pt) -- (320pt,383pt) -- (385.8595pt,383pt) arc (270:360:5pt) -- (390.8595pt,388pt) -- (390.8595pt,426pt) arc (0:90:5pt) -- (385.8595pt,431pt) -- (320pt,431pt) arc (90:180:5pt) -- (315pt,426pt) -- cycle;
\draw[pstyle4] (330pt,399pt) ellipse (11pt and 11pt);
\node at (330pt,399pt)[]{\textbf{\Large I}};
\node at (344pt,390.127pt)[below right,color=black]{\textit{Player}};
\draw[pstyle3] (316pt,415pt) -- (389.8595pt,415pt);
\draw[pstyle3] (316pt,423pt) -- (389.8595pt,423pt);
\draw[pstyle1] (426.5pt,388pt) arc (180:270:5pt) -- (431.5pt,383pt) -- (498.0474pt,383pt) arc (270:360:5pt) -- (503.0474pt,388pt) -- (503.0474pt,426pt) arc (0:90:5pt) -- (498.0474pt,431pt) -- (431.5pt,431pt) arc (90:180:5pt) -- (426.5pt,426pt) -- cycle;
\draw[pstyle4] (441.5pt,399pt) ellipse (11pt and 11pt);
\node at (441.5pt,399pt)[]{\textbf{\Large I}};
\node at (455.5pt,390.127pt)[below right,color=black]{\textit{Enemy}};
\draw[pstyle3] (427.5pt,415pt) -- (502.0474pt,415pt);
\draw[pstyle3] (427.5pt,423pt) -- (502.0474pt,423pt);
\draw[pstyle1] (539pt,388pt) arc (180:270:5pt) -- (544pt,383pt) -- (625.6pt,383pt) arc (270:360:5pt) -- (630.6pt,388pt) -- (630.6pt,426pt) arc (0:90:5pt) -- (625.6pt,431pt) -- (544pt,431pt) arc (90:180:5pt) -- (539pt,426pt) -- cycle;
\draw[pstyle4] (554pt,399pt) ellipse (11pt and 11pt);
\node at (554pt,399pt)[]{\textbf{\Large I}};
\node at (568pt,390.127pt)[below right,color=black]{\textit{Attribute}};
\draw[pstyle3] (540pt,415pt) -- (629.6pt,415pt);
\draw[pstyle3] (540pt,423pt) -- (629.6pt,423pt);
\draw[pstyle1] (322.5pt,637.5pt) arc (180:270:5pt) -- (327.5pt,632.5pt) -- (378.54pt,632.5pt) arc (270:360:5pt) -- (383.54pt,637.5pt) -- (383.54pt,675.5pt) arc (0:90:5pt) -- (378.54pt,680.5pt) -- (327.5pt,680.5pt) arc (90:180:5pt) -- (322.5pt,675.5pt) -- cycle;
\draw[pstyle4] (337.5pt,648.5pt) ellipse (11pt and 11pt);
\node at (337.5pt,648.5pt)[]{\textbf{\Large I}};
\node at (351.5pt,639.627pt)[below right,color=black]{\textit{Item}};
\draw[pstyle3] (323.5pt,664.5pt) -- (382.54pt,664.5pt);
\draw[pstyle3] (323.5pt,672.5pt) -- (382.54pt,672.5pt);
\draw[pstyle1] (305pt,526.5pt) arc (180:270:5pt) -- (310pt,521.5pt) -- (395.6148pt,521.5pt) arc (270:360:5pt) -- (400.6148pt,526.5pt) -- (400.6148pt,564.5pt) arc (0:90:5pt) -- (395.6148pt,569.5pt) -- (310pt,569.5pt) arc (90:180:5pt) -- (305pt,564.5pt) -- cycle;
\draw[pstyle4] (320pt,537.5pt) ellipse (11pt and 11pt);
\node at (320pt,537.5pt)[]{\textbf{\Large I}};
\node at (334pt,528.627pt)[below right,color=black]{\textit{Inventory}};
\draw[pstyle3] (306pt,553.5pt) -- (399.6148pt,553.5pt);
\draw[pstyle3] (306pt,561.5pt) -- (399.6148pt,561.5pt);
\draw[pstyle1] (265pt,91pt) arc (180:270:5pt) -- (270pt,86pt) -- (344.3455pt,86pt) arc (270:360:5pt) -- (349.3455pt,91pt) -- (349.3455pt,129pt) arc (0:90:5pt) -- (344.3455pt,134pt) -- (270pt,134pt) arc (90:180:5pt) -- (265pt,129pt) -- cycle;
\draw[pstyle4] (280pt,102pt) ellipse (11pt and 11pt);
\node at (280pt,102pt)[]{\textbf{\Large I}};
\node at (294pt,93.127pt)[below right,color=black]{\textit{Combat}};
\draw[pstyle3] (266pt,118pt) -- (348.3455pt,118pt);
\draw[pstyle3] (266pt,126pt) -- (348.3455pt,126pt);
\draw[pstyle1] (687pt,526.5pt) arc (180:270:5pt) -- (692pt,521.5pt) -- (794.1015pt,521.5pt) arc (270:360:5pt) -- (799.1015pt,526.5pt) -- (799.1015pt,564.5pt) arc (0:90:5pt) -- (794.1015pt,569.5pt) -- (692pt,569.5pt) arc (90:180:5pt) -- (687pt,564.5pt) -- cycle;
\draw[pstyle2] (702pt,537.5pt) ellipse (11pt and 11pt);
\node at (702pt,537.5pt)[]{\textbf{\Large E}};
\node at (716pt,528.627pt)[below right,color=black]{MagicMove};
\draw[pstyle3] (688pt,553.5pt) -- (798.1015pt,553.5pt);
\draw[pstyle3] (688pt,561.5pt) -- (798.1015pt,561.5pt);
\draw[pstyle1] (692pt,388pt) arc (180:270:5pt) -- (697pt,383pt) -- (776.6571pt,383pt) arc (270:360:5pt) -- (781.6571pt,388pt) -- (781.6571pt,426pt) arc (0:90:5pt) -- (776.6571pt,431pt) -- (697pt,431pt) arc (90:180:5pt) -- (692pt,426pt) -- cycle;
\draw[pstyle4] (707pt,399pt) ellipse (11pt and 11pt);
\node at (707pt,399pt)[]{\textbf{\Large I}};
\node at (721pt,390.127pt)[below right,color=black]{\textit{MoveSet}};
\draw[pstyle3] (693pt,415pt) -- (780.6571pt,415pt);
\draw[pstyle3] (693pt,423pt) -- (780.6571pt,423pt);
\draw[pstyle1] (89.5pt,224pt) arc (180:270:5pt) -- (94.5pt,219pt) -- (185.4695pt,219pt) arc (270:360:5pt) -- (190.4695pt,224pt) -- (190.4695pt,262pt) arc (0:90:5pt) -- (185.4695pt,267pt) -- (94.5pt,267pt) arc (90:180:5pt) -- (89.5pt,262pt) -- cycle;
\draw[pstyle4] (104.5pt,235pt) ellipse (11pt and 11pt);
\node at (104.5pt,235pt)[]{\textbf{\Large I}};
\node at (118.5pt,226.127pt)[below right,color=black]{\textit{GameMap}};
\draw[pstyle3] (90.5pt,251pt) -- (189.4695pt,251pt);
\draw[pstyle3] (90.5pt,259pt) -- (189.4695pt,259pt);
\draw[pstyle1] (46.5pt,388pt) arc (180:270:5pt) -- (51.5pt,383pt) -- (132.7pt,383pt) arc (270:360:5pt) -- (137.7pt,388pt) -- (137.7pt,426pt) arc (0:90:5pt) -- (132.7pt,431pt) -- (51.5pt,431pt) arc (90:180:5pt) -- (46.5pt,426pt) -- cycle;
\draw[pstyle2] (61.5pt,399pt) ellipse (11pt and 11pt);
\node at (61.5pt,399pt)[]{\textbf{\Large E}};
\node at (75.5pt,390.127pt)[below right,color=black]{CellType};
\draw[pstyle3] (47.5pt,415pt) -- (136.7pt,415pt);
\draw[pstyle3] (47.5pt,423pt) -- (136.7pt,423pt);
\draw[pstyle1] (173pt,382.5pt) arc (180:270:5pt) -- (178pt,377.5pt) -- (253.8222pt,377.5pt) arc (270:360:5pt) -- (258.8222pt,382.5pt) -- (258.8222pt,431.457pt) arc (0:90:5pt) -- (253.8222pt,436.457pt) -- (178pt,436.457pt) arc (90:180:5pt) -- (173pt,431.457pt) -- cycle;
\draw[color=plantucolor0002,fill=plantucolor0005,line width=1.0pt] (188pt,398.9785pt) ellipse (11pt and 11pt);
\node at (188pt,398.9785pt)[]{\textbf{\Large C}};
\node at (201.6589pt,382.5pt)[below right,color=black]{\textit{\guillemotleft Record\guillemotright }};
\node at (202pt,397.7109pt)[below right,color=black]{Position};
\draw[pstyle3] (174pt,420.457pt) -- (257.8222pt,420.457pt);
\draw[pstyle3] (174pt,428.457pt) -- (257.8222pt,428.457pt);
\draw[pstyle6] (438.715pt,282.0169pt) ..controls (417.425pt,312.8169pt) and (390.29pt,352.06pt) .. (369.02pt,382.83pt);
\draw[pstyle6] (448.95pt,267.21pt) -- (433.7793pt,278.6052pt) -- (443.6506pt,285.4286pt) -- (448.95pt,267.21pt) -- cycle;
\draw[pstyle6] (465pt,285.21pt) ..controls (465pt,316.01pt) and (465pt,352.06pt) .. (465pt,382.83pt);
\draw[pstyle6] (465pt,267.21pt) -- (459pt,285.21pt) -- (471pt,285.21pt) -- (465pt,267.21pt) -- cycle;
\draw[pstyle6] (353pt,443.16pt) ..controls (353pt,468.26pt) and (353pt,496.09pt) .. (353pt,521.23pt);
\draw[pstyle7] (353pt,431.16pt) -- (349pt,437.16pt) -- (353pt,443.16pt) -- (357pt,437.16pt) -- (353pt,431.16pt) -- cycle;
\draw[pstyle6] (489.3338pt,276.8519pt) ..controls (512.1538pt,307.6519pt) and (545.04pt,352.06pt) .. (567.84pt,382.83pt);
\draw[pstyle6] (482.19pt,267.21pt) -- (482.5479pt,274.4122pt) -- (489.3338pt,276.8519pt) -- (488.9759pt,269.6497pt) -- (482.19pt,267.21pt) -- cycle;
\draw[pstyle6] (353pt,581.62pt) ..controls (353pt,600.11pt) and (353pt,613.93pt) .. (353pt,632.42pt);
\draw[pstyle6] (353pt,569.62pt) -- (349pt,575.62pt) -- (353pt,581.62pt) -- (357pt,575.62pt) -- (353pt,569.62pt) -- cycle;
\draw[pstyle6] (344.1191pt,141.7776pt) ..controls (372.9191pt,165.6476pt) and (408.34pt,195.02pt) .. (437.13pt,218.89pt);
\draw[pstyle6] (334.88pt,134.12pt) -- (336.947pt,141.0285pt) -- (344.1191pt,141.7776pt) -- (342.0521pt,134.8691pt) -- (334.88pt,134.12pt) -- cycle;
\draw[pstyle6] (738.5454pt,443.1485pt) ..controls (739.6454pt,468.2485pt) and (740.87pt,496.09pt) .. (741.98pt,521.23pt);
\draw[pstyle6] (738.02pt,431.16pt) -- (734.2865pt,437.3294pt) -- (738.5454pt,443.1485pt) -- (742.2789pt,436.9791pt) -- (738.02pt,431.16pt) -- cycle;
\draw[pstyle6] (525.2466pt,272.3976pt) ..controls (562.5366pt,290.4876pt) and (603.68pt,311.81pt) .. (647pt,339pt) ..controls (668.28pt,352.36pt) and (690.97pt,369.27pt) .. (708.24pt,382.76pt);
\draw[pstyle7] (514.45pt,267.16pt) -- (518.1024pt,273.3777pt) -- (525.2466pt,272.3976pt) -- (521.5942pt,266.1799pt) -- (514.45pt,267.16pt) -- cycle;
\draw[color=plantucolor0002,line width=1.0pt,dash pattern=on 7.0pt off 7.0pt] (865.5723pt,435.1538pt) ..controls (837.3523pt,460.3238pt) and (797pt,496.33pt) .. (768.83pt,521.46pt);
\draw[pstyle7] (870.05pt,431.16pt) -- (860.6709pt,434.1655pt) -- (866.3186pt,434.4881pt) -- (865.9959pt,440.1358pt) -- (870.05pt,431.16pt) -- cycle;
\draw[pstyle6] (546.1008pt,255.8783pt) ..controls (619.0908pt,268.2683pt) and (724.02pt,291.86pt) .. (815pt,339pt) ..controls (836.88pt,350.33pt) and (858.2pt,368.35pt) .. (873.48pt,382.84pt);
\draw[pstyle7] (534.27pt,253.87pt) -- (539.516pt,258.8177pt) -- (546.1008pt,255.8783pt) -- (540.8548pt,250.9305pt) -- (534.27pt,253.87pt) -- cycle;
\node at (798pt,298pt)[below right,color=black]{Classe del };
\node at (799.9234pt,314.4785pt)[below right,color=black]{ character.};
\draw[pstyle6] (129.713pt,278.7162pt) ..controls (120.593pt,309.5162pt) and (107.98pt,352.06pt) .. (98.86pt,382.83pt);
\draw[pstyle6] (133.12pt,267.21pt) -- (127.5811pt,271.8274pt) -- (129.713pt,278.7162pt) -- (135.2519pt,274.0988pt) -- (133.12pt,267.21pt) -- cycle;
\draw[pstyle6] (155.9871pt,278.0737pt) ..controls (169.6171pt,307.1237pt) and (187.85pt,345.99pt) .. (202.57pt,377.37pt);
\draw[pstyle6] (150.89pt,267.21pt) -- (149.8173pt,274.3409pt) -- (155.9871pt,278.0737pt) -- (157.0598pt,270.9428pt) -- (150.89pt,267.21pt) -- cycle;
\draw[pstyle6] (409.5636pt,272.8004pt) ..controls (375.4836pt,290.9904pt) and (338.48pt,312.02pt) .. (299pt,339pt) ..controls (281.88pt,350.7pt) and (263.85pt,364.99pt) .. (248.99pt,377.36pt);
\draw[pstyle7] (420.15pt,267.15pt) -- (412.9733pt,266.4464pt) -- (409.5636pt,272.8004pt) -- (416.7403pt,273.504pt) -- (420.15pt,267.15pt) -- cycle;
\draw[pstyle6] (296.3221pt,145.6166pt) ..controls (279.8521pt,198.9966pt) and (242.8pt,319.13pt) .. (224.82pt,377.42pt);
\draw[pstyle7] (299.86pt,134.15pt) -- (294.2688pt,138.704pt) -- (296.3221pt,145.6166pt) -- (301.9132pt,141.0626pt) -- (299.86pt,134.15pt) -- cycle;
\end{tikzpicture}
}

	\caption{Schema UML del dominio.} \label{fig:Schema UML del dominio.}
\end{figure}

\end{document}